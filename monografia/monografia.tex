\documentclass[a4paper]{book}
\usepackage[brazilian]{babel}
\usepackage[utf8]{inputenc}
\usepackage[T1]{fontenc}
\usepackage{amsthm}
\usepackage[colorlinks=true, allcolors=blue]{hyperref}
\usepackage{titling}
\renewcommand\maketitlehooka{\null\mbox{}\vfill}
\renewcommand\maketitlehookd{\vfill\null}

\title{\textbf{Big-Line-Big-Clique Conjecture e Bloqueadores de Visibilidade}}
\author{Gabriel K. Lasso\\ Orientador: Carlos E. Ferreira}
\date{}

\begin{document}
\maketitle
\tableofcontents
\chapter{Introdução}

\section{Motivação}

\section{Estrutura do trabalho}

\chapter{Big-Line-Big-Clique conjecture}

\section{Caso $k=4$}
\subsection{Grafos de visibilidade planares}
Provar teorema 1 de \cite{planar}
\cite{visibilitygraph}
\section{Caso $k=5$}
\subsection{Erdös-Szekeres Theorem}
Provar teorema de Erdös-Szekers sobre polígonos convexos (Happy ending problem).
\cite{pentagon}
\section{Para conjunstos infinitos de pontos}
\cite{infinity}
\section{Dificuldades encontradas}
\subsection{The orchard problem e o grafo de Turán}
Problema citado em \cite{visblock}, solução do orchard problem tem menos arestas do que o grafo de Turán para $k\geq 5$

\subsection{Contuntos de pontos sem heptágonos vazios}
Construção de conjuntos arbitrariamente grandes de pontos sem heptágonos vazios de \cite{heptagon}

\chapter{Bloqueadores de visibilidade}

\section{Ordem de crescimento de $b(n)$}
\cite{block,blockers}

\section{Conjuntos de pontos em posição convexa}
\cite{block,blockers}


\bibliographystyle{plain}
\bibliography{../ref}{}

\end{document}

