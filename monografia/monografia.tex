\documentclass[a4paper]{book}
\usepackage[brazilian]{babel}
\usepackage[utf8]{inputenc}
\usepackage[T1]{fontenc}
\usepackage{amsthm}
\usepackage{amsmath}
\usepackage{amsfonts}
\usepackage{enumitem}
\usepackage{tikz}
\usetikzlibrary{positioning}
\usetikzlibrary{shapes}
\usepackage[colorlinks=true, allcolors=blue]{hyperref}

\title{\textbf{Big-Line-Big-Clique Conjecture e Bloqueadores de Visibilidade}}
\author{Gabriel K. Lasso\\ Orientador: Carlos E. Ferreira}
\date{}

\newtheorem{conjectura}{Conjectura}
\newtheorem{teorema}{Teorema}[section]
\newtheorem{corolario}{Corolário}[teorema]

\begin{document}
\maketitle
\tableofcontents
\chapter{Introdução}

\section{Motivação}
Aqui faremos um estudo de um problema de visibilidade que surgiu em 2005 e até o começo dessa década chamou a atenção de vários pesquisadores. Qual a quantidade máxima de pontos no plano de forma que não tenha conjuntos grandes de pontos colineares nem de pontos visíveis? A natureza desse problema lembra o problema de Ramsey, porém ele possui uma cara geométria e é extremamente difícil de lidar, tanto que não se sabe se existe essa quantidade máxima.

Se dois pontos não são visíveis, é porque um terceiro bloqueia a visibilidade deles. Outro problema estudado aqui é sobre o tamanho mínimo do conjunto de bloqueadores para um conjunto de pontos.

Todos esses conceitos serão definidos com mais formalidade mais pra frente. O primeiro problema é conhecido como \textbf{conjectura big-line-big-clique} e o segundo é uma questão natural sobre \textbf{bloqueadores de visibilidade}.


\section{Estrutura do trabalho}
No caítulo 2 vamos definir alguns conceitos que serão base para os demais, assim como demonstrar algumas propriedades chaves desses conceitos. No capítulo 3 vamos mostrar os resultados conhecidos sobre a conjectura big-line-big-clique, mostrar algumas dificuldades encontradas na tentativa de resolver esse problema e discutir alguns fatos sobre ele. No capítulo 4 vamos focar no problema dos bloqueadores de visibilidade, mostrando cotas superiores e inferiores conhecidas e discutindo algumas coisas em aberto que as pessoas acreditam que sejam verade.

\chapter{Noções preliminares}
\section{Grafos}
Um grafo é um par $G=(V,E)$ em que $V$ é um conjunto enumerável (geralmente finito) e $E$ é um conjunto de pares não ordenados de elementos de $V$. Os elementos de $V$ são chamados \textbf{vértices}, e os elementos de $E$ são chamados \textbf{arestas}.

Se $G$ é um grafo, chamamos de $V(G)$ seu conjunto de vértices e de $E(G)$ seu conjunto de arestas.

Chamamos \textbf{ordem} de um grafo a cardinalidade de seu conjunto de vértices.

Se uma aresta $e={u,v}\in E(G)$, dizemos que $u$ e $v$ são \textbf{vizinhos} ou \textbf{adjacentes}.

Um grafo \textbf{completo} é um grafo em que todos dois vértices distintos são adjacentes. Chamamos o grafo completo de ordem $n$ de $K_n$.

Se $G$ é um grafo e  $S\subset V(G)$ é um conjunto de vértices, chamamos de \textbf{subgrafo} de $G$ induzido por $S$ o grafo $G[S]=(S,E_S)$, em que $E_S=\{{u,v}\in E(G)|u,v\in S\}$. Nesse caso, dizemos que $G$ contém uma cópia de $G[S]$ ou simplesmente que $G$ contém $G[S]$, ou ainda que $G$ possiu $G[S]$.

Um \textbf{k-clique} é um conjunto de vértices $C$ tal que $G[C]$ é um grafo completo de ordem k.

\section{Grafo de visibilidade}
Na sessão anterior, definimos conceitos básicos de teoria dos grafos. Aqui vamos estudar um caso bem particular.

Seja $P$ é um conjunto de pontos no plano tal que todo ponto de $P$ é isolado e $x,y\in P$. Dizemos que $x$ e $y$ são visíveis em $P$ se $\overline{xy}\cap P =\{x,y\}$. $p_1, p_2, ..., p_n$ são visíveis se são visíveis dois a dois.

O grafo de visibilidade de um conjunto $P$ de pontos no plano é o grafo construído da seguinte forma: o conjunto de vértices é $P$ e há uma aresta entre dois vértices $x$ e $y$ se $x$ e $y$ são visíveis. Denotamos esse grafo por $\mathcal V(P)$ e se ${p,q}$ é uma aresta, às vezes nos referimos a ela como o segmento aberto $\overline{pq}$.

Dizemos que um grafo de visibilidade é planar se suas arestas não se interceptarem.

\begin{teorema}\cite{visibility}\label{planork4}
    Todo grafo de visibilidade finito é planar ou contém uma cópia de $K_4$.
\end{teorema}
\begin{proof}
    Seja $G=\mathcal V(P)$ um grafo de visibilidade finito. Suponha que $G$ não seja planar. Vamos mostrar que $G$ contém $K_4$.

    Como $G$ não é planar, existe alguma aresta $\overline{ab}$ que intercepta com outras (pelo menos uma outra). A reta $\overleftrightarrow{ab}$ separa o plano em dois semiplanos $D$ e $E$. Seja $p$ o ponto de $D$ mais próximo do segmento $\overline{ab}$ e $q$ o ponto de $E$ mais próximo do segmento $\overline{ab}$. Pela escolha de $p$ e $q$, os pontos $a$, $b$, $p$ e $q$ são visíveis, logo $G[\{a,b,p,q\}]$ é um $K_4$.
\end{proof}

\section{Geometria combinatória}
Vamos dar algumas definições a respeito de conjuntos de pontos no plano. Seja $P$ um conjunto finito de pontos no plano e $conv(P)$ o casco convexo de $P$. Dizemos que $P$ está em \textbf{posição geral} se não existirem três pontos colineares em $P$. 

Dizemos que $P$ está em \textbf{posição convexa} se todo ponto de $P$ estiver na fronteira de $conv(P)$.

Um ponto $v\in P$ é uma \textbf{ponta} de $P$ se $conv(P)\neq conv(P\backslash v)$.

Se todos os pontos de $P$ forem pontas, dizemos que $P$ está em \textbf{posição estritamente convexa}.

Um \textbf{$k$-ágono} (respectivamente um \textbf{$k$-ágono estritamente convexo}) é o casco convexo de um conjunto de $k$ pontos (respectivamente em posição estritamente convexa).

Se $X$ for um subconjunto de $P$ de $k$ pontos em posição estritamente convexa tal que $conv(X)\cap P=X$, então  $conv(X)$ é um \textbf{$k$-buraco} ou um \textbf{$k$-ágono vazio estritamente convexo} ou simplesmente um \textbf{$k$-ágono vazio}.

Chamamos $3$-ágonos de triângulos, $4$-ágonos de quadriláteros, $5$-ágonos de pentágonos, etc.

\subsection{Teorema de Erdös-Szekeres}
Provar teorema de Erdös-Szekers sobre polígonos convexos (Happy ending problem).

\subsection{Teorema de Sylvester-Gallai}
Todo contunto finito de pontos tem todos os pontos colineares ou existe uma reta que passa por exatamente dois desses pontos.

\chapter{Big-Line-Big-Clique conjecture}
A conjectura seguinte foi proposta por \cite{visibilitygraph} em 2005, e ainda não se sabe se ela vale para valores a partir de  $k=6$ e $l=4$.
\begin{conjectura}\label{conj1}
    Dados dois inteiros $k,l\geq2$, existe um $n=n(k,l)$ tal que todo conjunto finito $P$ com pelo menos $n$ pontos no plano contém $k$ pontos visíveis (alternativamente, $\mathcal V(P)$ possui um $k$-clique) ou $l$ pontos colineares.
\end{conjectura}


Para nos habituarmos com o problema, vamos começar pelos casos mais básicos:

\section{Casos triviais}
Primeiramente, se $k=2$ ou $l=2$, com certeza todo conjunto com pelo menos dois pontos tem dois pontos colineares e dois pontos visíveis.

Se $k=3$, ou se tem três pontos visíveis ou todos os pontos no conjunto são colineares, então $n(3,l)=max\{3,l\}$.

Se $l=3$, ou todos os pontos são visíveis para todos os outros ou algum ponto não deixa outros dois se verem, se tendo três pontos colineares. $n(k,3) = max\{k,3\}$.


\section{Caso $k=4$}
Vamos mostrar que os conjuntos que não possuem quatro pontos visíveis possuem muitos pontos colineares. Isto é, para todo $l$ existe um $n$ tal que todo conjunto $P$ com pelo menos $n$ pontos que não possui quatro pontos visívels possui pelo menos $l$ pontos colineares.

Pelo teorema \ref{planork4}, um conjunto que não possui quatro pontos visíveis possui grafo de visibilidade planar. O teorema a seguir caracteriza os grafos de visibilidade planares:

\begin{figure}
    \begin{tikzpicture}
        \node(A) {$(a)$};
        \node[draw, shape=circle] (A1) [right = 0.3cm of A] {};
        \node[draw, shape=circle] (A2) [right = 0.6cm of A1] {};
        \node[draw, shape=circle] (A3) [right = 0.6cm of A2] {};
        \node[draw, shape=circle] (A4) [right = 0.6cm of A3] {};
        \node[draw, shape=circle] (A5) [right = 0.6cm of A4] {};
        \draw (A1)--(A2);
        \draw (A2)--(A3);
        \draw (A3)--(A4);
        \draw (A4)--(A5);

        \node(B) [below = 1.2cm of A] {$(b)$};
        \node[draw, shape=circle] (B1) [right = 0.3cm of B] {};
        \node[draw, shape=circle] (B2) [right = 0.6cm of B1] {};
        \node[draw, shape=circle] (B3) [right = 0.6cm of B2] {};
        \node[draw, shape=circle] (B4) [right = 0.6cm of B3] {};
        \node[draw, shape=circle] (B5) [right = 0.6cm of B4] {};
        \node[draw, shape=circle] (B6) [above = 0.6cm of B3] {};
        \draw (B1)--(B2);
        \draw (B2)--(B3);
        \draw (B3)--(B4);
        \draw (B4)--(B5);
        \draw (B1)--(B6);
        \draw (B2)--(B6);
        \draw (B3)--(B6);
        \draw (B4)--(B6);
        \draw (B5)--(B6);

        \node(C) [below = 1.2cm of B] {$(c)$};
        \node[draw, shape=circle] (C1) [right = 0.3cm of C] {};
        \node[draw, shape=circle] (C2) [right = 0.6cm of C1] {};
        \node[draw, shape=circle] (C3) [right = 0.6cm of C2] {};
        \node[draw, shape=circle] (C4) [right = 0.6cm of C3] {};
        \node[draw, shape=circle] (C5) [right = 0.6cm of C4] {};
        \node[draw, shape=circle] (C6) [above = 0.6cm of C3] {};
        \node[draw, shape=circle] (C7) [below = 0.6cm of C3] {};
        \draw (C1)--(C2);
        \draw (C2)--(C3);
        \draw (C3)--(C4);
        \draw (C4)--(C5);
        \draw (C1)--(C6);
        \draw (C2)--(C6);
        \draw (C3)--(C6);
        \draw (C4)--(C6);
        \draw (C5)--(C6);
        \draw (C1)--(C7);
        \draw (C2)--(C7);
        \draw (C3)--(C7);
        \draw (C4)--(C7);
        \draw (C5)--(C7);

        \node(D) [right = 5.4cm of A] {$(d)$};
        \node (D') [right = 0.3cm of D] {};
        \node[draw, shape=circle] (D1) [right = 0.3cm of D'] {};
        \node[draw, shape=circle] (D2) [right = 0.6cm of D1] {};
        \node[draw, shape=circle] (D3) [right = 0.6cm of D2] {};
        \node[draw, shape=circle] (D4) [right = 0.6cm of D3] {};
        \node[draw, shape=circle] (D5) [right = 0.6cm of D4] {};
        \node[draw, shape=circle] (D6) [above = 0.6cm of D'] {};
        \node[draw, shape=circle] (D7) [below = 0.6cm of D'] {};
        \draw (D1)--(D2);
        \draw (D2)--(D3);
        \draw (D3)--(D4);
        \draw (D4)--(D5);
        \draw (D1)--(D6);
        \draw (D2)--(D6);
        \draw (D3)--(D6);
        \draw (D4)--(D6);
        \draw (D5)--(D6);
        \draw (D1)--(D7);
        \draw (D2)--(D7);
        \draw (D3)--(D7);
        \draw (D4)--(D7);
        \draw (D5)--(D7);
        \draw (D6)--(D7);

        \node(E) [below =2.4cm of D] {$(e)$};
        \node[draw, shape=circle] (E1) [below right = 1cm and 0.3cm of E] {};
        \node[draw, shape=circle] (E2) [right = 3cm of E1] {};
        \node[draw, shape=circle] (E3) [above right = 2.8cm and 1.5cm of E1] {};
        \node[draw, shape=circle] (E4) [below left = 1.1cm and 0cm of E3] {};
        \node[draw, shape=circle] (E5) [below left = 0.9cm and 0cm of E4] {};
        \node[draw, shape=circle] (E6) [below right = 0.5cm and 0.55cm of E4] {};
        \draw (E1)--(E2);
        \draw (E2)--(E3);
        \draw (E3)--(E1);
        \draw (E1)--(E4);
        \draw (E1)--(E5);
        \draw (E2)--(E5);
        \draw (E2)--(E6);
        \draw (E3)--(E6);
        \draw (E3)--(E4);
        \draw (E4)--(E5);
        \draw (E5)--(E6);
        \draw (E6)--(E4);
    \end{tikzpicture}
    \caption{Tipos de grafos de visibilidade planares}
    \label{fig:visibilidadeplanar}
\end{figure}

\begin{teorema}\cite{planar}\label{visibilidadeplanar}
    Seja $P$ um conjunto finito de pontos no plano. Então $\mathcal V(P)$ é planar se e somente se satisfaz uma das condições:
    \begin{enumerate}[label=(\alph*)]
        \item
            Todos os pontos de $P$ são colineares.
        \item
            Todos os pontos de $P$ são colineares exceto um.
        \item
            Todos os pontos de $P$ são colineares exceto dois pontos não visíveis.
        \item
            Todos os pontos de $P$ são colieares exceto dois pontos $p$ e $q$ tais que o segmento $\overline{pq}$ não intercepta o menor segmento que contém $P\setminus\{p,q\}$
        \item
            $\mathcal V(P)$ é o grafo $(e)$ desenhado na figura \ref{fig:visibilidadeplanar}
    \end{enumerate}
\end{teorema}
\begin{proof}
    Como pode ser visto na figura \ref{fig:visibilidadeplanar} e pode ser facilmente demonstrado, se um grafo de visibilidade satisfaz alguma das condições $(a)-(e)$, ele é planar. Para mostrar a outra direção, seja $P$ um conjunto finito de pontos no plano tal que $\mathcal V(P)$ é planar.

    Se $P$ contém no máximo dois pontos, $P$ satisfaz $(a)$. Então vamos supor que $|P|\geq 3$. Seja $L$ um maior conjunto de pontos colineares em $P$ e $\hat L$ a reta que contém $L$. $|L|\geq 2$. Se $|L|=2$, então $L$ satisfaz $(a)$, $(b)$ ou $(e)$. Vamos supor então que $|L|\geq3$.

    A reta $\hat L$ divide o plano em dois semiplanos. Sejam $S$ e $T$ os conjuntos de vértices em cada um desses semiplanos tal que $|S|\geq|T|$. Se $|S|$ for no máximo 1, então $P$ satisfaz algum de $(a)-(d)$. Então supor que $|S|\geq2$. Seja $v$ um ponto de $S$ mais próximo de $\hat L$ e seja $w$ um ponto de $S\setminus\{v\}$ mais próximo de $\hat L$. Note que a reta que contém $v$ e $w$ não é paralela a $\hat L$, pois, se fosse, para todos $x,y\in L$, as arestas $\overline{vx}$ e $\overline{wy}$ ou as arestas $\overline{vy}$ e $\overline{wx}$ se interceptariam e, como o grafo é planar, existiria um vértice nessa interseção que seria mais próximo de $\hat L$ do que $v$, contradizendo a escolha de $v$. Então a reta que contém $v$ e $w$ intercepta $\hat L$ em um ponto, digamos, $p$.

    Vamos mostrar que em $\hat L$ se tem no máximo um vértice de cada lado de $p$. Para isso, suponha que existam dois vértices $x,y\in L$ tais que $x$ esteja entre $y$ e $p$. Então as arestas $\overline{vy}$ e $\overline{wx}$ se interceptam num ponto mais próximo de $\hat L$ do que $v$ e, como o grafo é planar, deve existir um vértice nessa interseção, contradizendo a escolha de $v$. Mas como $|L|\geq3$, $L$ tem que possuir exatamente $3$ pontos: $p$ e um ponto de cada lado de $p$, que vamos chamar de $x$ e $y$.

    Agora vamos mostrar que $S=\{v,w\}$. Suponha que $S$ contém um terceiro ponto $u$. $u$ não está na mesma reta que $p$, $v$ e $w$ pois $L$ tem tamanho máximo. Pela escolha de $v$ e $w$, $u$ está mais longe de $\hat L$ do que $w$. Então $\overline{uv}$ intercepta $\overline{wx}$ ou $\overline{wy}$ num ponto mais próximo de $\hat L$ do que $w$, contradizendo a escolha de $w$. Logo $|S|=2$. Então $|T|\leq2$.

    Se $T=\emptyset$, então $P$ satisfaz $(c)$. Suponha que $T\neq\emptyset$. Seja $u$ um ponto de $T$ e $q$ a interseção da reta que contém $u$ e $v$ e da reta $\hat L$. Suponha que $q$ não seja um vértice de $L$. Se $q$ está entre $x$ e $y$, então as arestas $\overline{vu}$ e $\overline{px}$ ou $\overline{py}$ se interceptam em $q$, e como o grafo é planar, $q$ tem que ser um vértice do grafo, o que é uma contradição. Similarmente, se $q$ não está entre $x$ e $y$, as arestas $\overline{uv}$ e $\overline{wx}$ ou $\overline{wy}$ se interceptam no ponto $q$, gerando uma contradição. Então $u$ intercepta $\hat L$ num vértice. Esse vértice não pode ser $p$, pois se fosse isso contradiria a escolha de $L$.

    Suponha que existam dois pontos $u_1$ e $u_2$ em $T$. Se $u_1$ está na mesma reta que $v$ e $x$ e $u_2$ está na mesma reta que $v$ e $y$. As arestas $\overline{u_1y}$ e $\overline{u_2x}$ se interceptam em um ponto de $T$ que não está na mesma reta que $v$ e $x$ nem que $v$ e $y$, o que é uma contradição. Se $u_1$ e $u_2$ estão na mesma reta que $v$ e $x$, suponha sem perda de generalidade que $u_1$ está entre $x$ e $u_2$, então as arestas $\overline{u_1y}$ e $\overline{u_2x}$ se interceptam em um ponto de $T$ que não está na mesma reta que $v$ e $x$ nem que $v$ e $y$, o que é uma contradição. Do mesmo jeito, se $u_1$ e $u_2$ estão na mesma reta que $v$ e $y$ também temos uma contradição. Portanto $|T|=1$. Sem perda de generalidade, digamos que $u$ é colinear com $v$ e $x$. Assim $\{p,u,v,x,y,w\}$ forma o grafo $(e)$ da figura \ref{fig:visibilidadeplanar}
\end{proof}
\begin{corolario}
    Seja $P$ um conjunto finito de pontos no plano. As seguintes afirmações são equivalentes:
    \begin{itemize}
        \item
            $P$ satisfaz $(a)$, $(b)$, $(c)$ ou $(e)$ do teorema \ref{visibilidadeplanar}.
        \item
            $\mathcal V(P)$ não tem $K_4$.
    \end{itemize}
\end{corolario}
\begin{proof}
    Se $P$ satisfaz $(a)$, $(b)$, $(c)$ ou $(e)$ do teorema \ref{visibilidadeplanar}, é fácil ver que $\mathcal V(P)$ não tem $K_4$.

    Se $\mathcal V(P)$ não tem $K_4$, pelo teorema \ref{planork4}, $\mathcal V(P)$ é planar. Logo, pelo teorema \ref{visibilidadeplanar}, $P$ satisfaz $(a)$, $(b)$, $(c)$, $(d)$ ou $(e)$. Mas como $(d)$ possiu $K_4$, $P$ satisfaz $(a)$, $(b)$, $(c)$ ou $(e)$.
\end{proof}

\begin{corolario}
    A conjectura \ref{conj1} é verdadeira para $k=4$ com $n(4,l)=max\{7,l+2\}$ se $l>3$.
\end{corolario}

\section{Caso $k=5$}
\subsection{Pontos em posição convexa}
\cite{pentagon}

\section{Para conjunstos infinitos de pontos}
Em 2010\cite{infinity}, Atila Pór e David Wood mostraram que a conjectura \ref{conj1} não vale para conjuntos infinitos de pontos com a seguinte construção:

\begin{teorema}
    Existem conjuntos infinitos enumeráveis de pontos sem 4 pontos colineares e sem 3 pontos visíveis dois a dois.
\end{teorema}
\begin{proof}
    Vamos construir indutivamente um conjunto de pontos com tal propriedade.

    Sejam $x_1$, $x_2$ e $x_3$ três pontos não colineares no plano.

    Dados pontos $x_1,...,x_{n-1}$ não conlineares, vamos definir o ponto $x_n$ da seguinte forma: pelo teorema de Sylvester-Gallai, existe uma reta que passa por exatamente dois pontos de $x_1,...,x_{n-1}$. Seja $\mathcal L=\overleftrightarrow{x_ix_j}$ a reta que passa or exatamente dois pontos de $x_1,...,x_{n-1}$ com $i<j$ e com i mínimo entre as com j mínimo. Coloque $x_n$ no segmento $\overline{x_ix_j}$ tal que ela seja a única reta que contenha $x_n$ e mais dois pontos (isso é possível, já que somente uma quantidade finita de pontos de $\mathcal L$ não pode ser escolhida).

    Vamos chamar de $P = \{x_i|i\in \mathbb N\}$. Por construção, $P$ não tem 4 pontos colineares e se $x_i$ e $x_k$ com $i<k$ são visíveis então existe um outro ponto colinear $x_j$ com $j<k$ e $x_k$ está no segmento $\overline{x_ix_j}$.

    Suponha que $x_i$, $x_j$ e $x_k$ são dois a dois visíveis com $i<j<k$. Então existem pontos $x_{i'}$ e $x_{j'}$ com $i'<k$ e $j'<k$ tais que $x_k$ está no segmento $\overline{x_ix_{i'}}$ e no segmento $\overline{x_jx_{j'}}$. Pela escolha de $x_k$, só existe uma reta que contém $x_k$ e mais dois pontos entre $x_1,...x_{k-1}$. Então $i=j'$ e $j=i'$. Mas então $P\cap\overline{x_ix_j}=\{x_i,x_j,x_k\}$, ou seja, $x_i$ e $x_j$ não são visíveis, ou seja, $x_i$ e $x_j$ não são visíveis.

    Portanto não existem 3 pontos visíveis dois a dois em $P$.

\end{proof}

\section{Dificuldades encontradas}
\subsection{The orchard problem e o grafo de Turán}
Problema citado em \cite{visblock}, solução do orchard problem tem menos arestas do que o grafo de Turán para $k\geq 5$

\subsection{Contuntos de pontos sem heptágonos vazios}
Dado que o teorema de Erdös-Szekeres vale, Erdös colocou o problema de determinar se para um dado $n$ existe um $g(n)$ tal que todo conjunto com $g(n)$ pontos em posição geral contém um $n$-ágono vazio.

Se essa questão fosse respondida afirmativamente, a conjectura \ref{conj1} sairia como corolário.

Para $n=3$ e $n=4$ é fácil ver que $g(3)=3$ e $g(4)=5$. Harborth\cite{Harborth1978} provou que $g(5)=10$ e Gerken\cite{Gerken} e Nicolas\cite{Nicolas} provaram que $g(6)$ existe.

No entanto, Horton\cite{heptagon} mostrou uma constrção de conjuntos arbitrariamente grandes sem heptágonos vazios.

\begin{teorema}
Existem conjuntos arbitrariamente grandes sem heptágonos.
\end{teorema}
\begin{proof}

\end{proof}
\chapter{Bloqueadores de visibilidade}

\section{Ordem de crescimento de $b(n)$}
\cite{block,blockers}

\section{Conjuntos de pontos em posição convexa}
\cite{block,blockers}


\bibliographystyle{plain}
\bibliography{../ref}{}

\end{document}

