\documentclass[a4paper]{book}
\usepackage[brazilian]{babel}
\usepackage[utf8]{inputenc}
\usepackage[T1]{fontenc}
\usepackage{amsthm}
\usepackage{mathtools}
\usepackage{amsmath}
\usepackage{amsfonts}
\usepackage{amssymb}
\usepackage{enumitem}
\usepackage{tikz}
\usetikzlibrary{positioning}
\usetikzlibrary{shapes}
\usepackage[colorlinks=true, allcolors=blue]{hyperref}

\DeclarePairedDelimiter\ceil{\lceil}{\rceil}
\DeclarePairedDelimiter\floor{\lfloor}{\rfloor}

\title{\textbf{Big-Line-Big-Clique Conjecture e Bloqueadores de Visibilidade}}
\author{Gabriel K. Lasso\\ Orientador: Carlos E. Ferreira}
\date{}

\newtheorem{conjectura}{Conjectura}
\newtheorem{fato}{Fato}
\newtheorem{prop}{Proposição}
\newtheorem{problem}{Problema}
\newtheorem{lema}{Lema}
\newtheorem{teorema}{Teorema}
\newtheorem{corolario}{Corolário}[teorema]

\begin{document}
\maketitle
\tableofcontents
\chapter{Introdução}

\section{Motivação}
A teoria de Ramsey é um campo da combinatória que estuda questôes como "quão grande uma estrutura deve ser para que ela possua uma subestrutura com uma certa propriedade?". Um exemplo clássico é o problema da festa: quantas pessoas tem que ter em uma festa para que nela existam ou três pessoas que se conhecem ou três pessoas que não se conhecem entre si. Nesse caso sabe-se que bastam 6 pessoas. Essa teoria possui aplicações em diversos outros campos da matemática, como a topologia, a teoria ergódica e a geometria.\cite{ES}

Aqui fazemos um estudo de um problema geométrico ligado a teoria de Ramsey. Tal problema foi colocado em 2005 por Jan Kára, Attila Pór e David R. Wood enquanto estudavam o número cromático de grafos de visibilidade. O problema pode ser colocado da seguinte forma: "Quão grande deve ser um conjunto de pontos para que ele conhtenhai ou $l$ pontos colineares ou $k$ pontos visíveis dois a dois?".

Uma observação sobre esse problema é que se dois pontos não são visíveis, então existe um terceiro que bloqueia a visibilidade deles. Outro problema estudado aqui é sobre o tamanho mínimo do conjunto de bloqueadores para um conjunto de pontos.

Todos esses conceitos serão definidos com mais formalidade mais pra frente. O primeiro problema é conhecido como \textbf{conjectura big-line-big-clique} e o segundo é uma questão natural sobre \textbf{bloqueadores de visibilidade}.

\section{Estrutura do trabalho}
No capítulo 2 vamos definir alguns conceitos que serão base para os demais, assim como demonstrar algumas propriedades chaves desses conceitos. No capítulo 3 vamos mostrar os resultados conhecidos na literatura sobre a conjectura big-line-big-clique, mostrar algumas dificuldades encontradas na tentativa de resolver esse problema e discutir alguns fatos sobre ele. No capítulo 4 vamos focar no problema dos bloqueadores de visibilidade, mostrando cotas superiores e inferiores conhecidas e discutindo algumas coisas em aberto que as pessoas acreditam que sejam verdade.

\chapter{Noções preliminares}
\section{Convenções}
Vamos começar definindo algumas convenções de notação:
\begin{itemize}
    \item $|X|$ é a cardinalidade do conjunto $X$.
    \item $[n]=\{1,...,n\}$
    \item $A\setminus B = \{a\in A|a\notin B\}$
    \item Um $k$-conjunto é um conjunto de cardinalidade $k$ (e um $k$-subconjunto é um subconjuto de cardinalidade $k$)
    \item $\overline{pq}$ é o segmento de reta com extremos em $p$ e $q$.
    \item $\overrightarrow{pq}$ é o segmento de reta orientado com origem em $p$ e extremidade em $q$.
    \item $\overleftrightarrow{pq}$ é a reta que passa pelos pontos $p$ e $q$.
    \item $\Delta(a,b,c)=\{\alpha a+\beta b+\gamma c|\alpha+\beta+\gamma=1, \alpha>0, \beta>0, \gamma>0\}$ é o triângulo aberto com vértices $a$, $b$ e $c$.
    \item $\Delta[a,b,c]=\{\alpha a+\beta b+\gamma c|\alpha+\beta+\gamma=1, \alpha\geq0, \beta\geq0, \gamma\geq0\}$ é o triângulo fechado com vértices $a$, $b$ e $c$.
\end{itemize}

\section{Grafos}
Um grafo é um par $G=(V,E)$ em que $V$ é um conjunto enumerável (geralmente finito) e $E$ é um conjunto de pares não ordenados de elementos de $V$. Os elementos de $V$ são chamados \textbf{vértices}, e os elementos de $E$ são chamados \textbf{arestas}.

Se $G$ é um grafo, chamamos de $V(G)$ seu conjunto de vértices e de $E(G)$ seu conjunto de arestas.

Chamamos \textbf{ordem} de um grafo a cardinalidade de seu conjunto de vértices.

Se uma aresta $e=\{u,v\}\in E(G)$, dizemos que $u$ e $v$ são \textbf{vizinhos} ou \textbf{adjacentes}.

Um grafo \textbf{completo} é um grafo em que todos dois vértices distintos são adjacentes. Chamamos o grafo completo de ordem $n$ de $K_n$.

Se $G$ é um grafo e  $S\subset V(G)$ é um conjunto de vértices, chamamos de \textbf{subgrafo} de $G$ induzido por $S$ o grafo $G[S]=(S,E_S)$, em que $E_S=\{{u,v}\in E(G)|u,v\in S\}$. Nesse caso, dizemos que $G$ contém uma cópia de $G[S]$ ou simplesmente que $G$ contém $G[S]$, ou ainda que $G$ possui $G[S]$.

Uma \textbf{$k$-clique} é um conjunto de vértices $C$ tal que $G[C]$ é um grafo completo de ordem $k$.

Um grafo é \textbf{planar} se pode ser desenhado no plano sem que as curvas que representam as arestas se cruzem. Tal desenho é chamado de \textbf{imersão do grafo no plano}. Um grafo \textbf{plano} é um grafo planar com uma imersão no plano.

\section{Grafo de visibilidade}
Na sessão anterior, definimos conceitos básicos de teoria dos grafos. Aqui vamos estudar um caso bem particular.

Seja $P$ um conjunto de pontos no plano e $x,y\in P$. Dizemos que $x$ e $y$ são visíveis em $P$ se $\overline{xy}\cap P =\{x,y\}$. $p_1, p_2, ..., p_n$ são visíveis se são visíveis dois a dois.\footnote{Vamos considerar só conjuntos que são densos em nenhum lugar, i.e. o interior de $P$ é vazio.}

O grafo de visibilidade de um conjunto $P$ de pontos no plano é o grafo construído da seguinte forma: o conjunto de vértices é $P$ e há uma aresta entre dois vértices $x$ e $y$ se $x$ e $y$ são visíveis. Denotamos esse grafo por $\mathcal V(P)$ e se ${p,q}$ é uma aresta, às vezes nos referimos a ela como o segmento aberto $\overline{pq}$.

Dizemos que um grafo de visibilidade é plano se suas arestas não se interceptam.

\begin{teorema}\cite{visibility}\label{planork4}
    Todo grafo de visibilidade finito é planar ou contém uma cópia de $K_4$.
\end{teorema}
\begin{proof}
    Seja $G=\mathcal V(P)$ um grafo de visibilidade finito. Suponha que $G$ não seja planar. Vamos mostrar que $G$ contém $K_4$.

    Como $G$ não é planar, existe alguma aresta $\overline{ab}$ que intercepta com outras (pelo menos uma outra). A reta $\overleftrightarrow{ab}$ separa o plano em dois semiplanos $D$ e $E$. Seja $p$ o ponto de $D$ mais próximo do segmento $\overline{ab}$ e $q$ o ponto de $E$ mais próximo do segmento $\overline{ab}$ tal que $\overline{pq}$ intercepta $\overline{ab}$. Pela escolha de $p$ e $q$, os pontos $a$, $b$, $p$ e $q$ são visíveis, logo $G[\{a,b,p,q\}]$ é um $K_4$.
    \begin{center}
        \begin{tikzpicture}
            \fill (0,0.2) circle (1pt);
            \fill (0,1.8) circle (1pt);
            \fill (0.3,0.9) circle (1pt);
            \fill (-0.4,1) circle (1pt);
            \draw[dotted] (0,0.2) -- (0,1.8) -- (0.3,0.9) -- (-0.4,1) -- (0,0.2);
            \draw [dotted] (0,0.2) -- (0.3,0.9);
            \draw [dotted] (0,1.8) -- (-0.4,1);
            \node (a) at (0,0) {a};
            \node (b) at (0,2) {b};
            \node (p) at (0.5,1) {p};
            \node (q) at (-0.5,0.7) {q};
        \end{tikzpicture}
    \end{center}
\end{proof}

\section {Teoria de Ramsey}

\subsection {Problema da festa}
Um bom problema para ilustrara a teoria de Ramsey é o problema da festa já mencionado:

\begin{problem}
    Qual a quantidade mínima de pessoas que deve haver em uma festa para que três delas se conheçam ou três delas não se conheçam?
\end{problem}

Vamos resolver esse problema usando teoria dos grafos.

Pessoas serão representadas por vértices e a relação entre elas por arestas coloridas. Se duas pessoas se conhecem, vamos colorir a aresta entre elas de azul, e se duas pessoas não se conhecem, vamos colorir a aresta entre elas de vermelho.

Reformulando a pergunta com essa formulação, temos:

\begin{problem}
    Qual o menor $n$ tal que toda coloração de arestas\footnote{Uma coloração de arestas de um grafo $G$ é uma função de $E(G)$ em um conjunto de cores, digamos $\{Azul, Vermelho\}$} de $K_n$ em $\{azul, vermelho\}$ possui uma 3-clique azul ou uma 3-clique vermelha (i.e. uma 3-clique com todas as arestas que ligam seus vértices azuis/vermelhas)?
\end{problem}

Vamos primeiro mostrar que $n=6$ resolve o problema e depois mostrar que $n=5$ não resolve, concluindo que $n=6$ é o mínimo.

\begin{prop}
    Toda coloração das arestas de $K_6$ possui uma 3-clique azul ou uma 3-clique vermelha.
\end{prop}
\begin{proof}
    Seja $x$ um vértice de $K_6$. Existem $5$ arestas com extremidade em $x$. Por contagem, pelo menos $3$ dessas arestas são da mesma cor, digamos, vermelhas. Sejam $a,b,c$ vértices tais que $xa,xb,xc$ são arestas coloridas de vermelho.

    Vamos olhar para as arestas $ab,bc,ca$. Se alguma delas for vermelha, temos um triângulo vermelho. Se nenhuma delas for vermelha, temos um triângulo azul. Isso prova que $n=6$ é sulficiente para resulver o problema.
\end{proof}

Para a seguinte coloração de $K_5$, não temos nenhuma 3-clique monocromática, logo $n=6$ é a solução desejada para o problema da festa.
\begin{center}
    \begin{tikzpicture}
        \node[fill, circle] (V1) at (0,1.5) {};
        \node[fill, circle] (V2) at (1.43,0.46) {};
        \node[fill, circle] (V3) at (0.88,-1.21) {};
        \node[fill, circle] (V4) at (-0.88,-1.21) {};
        \node[fill, circle] (V5) at (-1.43,0.46) {};

        \draw[thin, color=blue] (V1) -- (V2) -- (V3) -- (V4) -- (V5) -- (V1);
        \draw[thin, red] (V1) -- (V3) -- (V5) -- (V2) -- (V4) -- (V1);
    \end{tikzpicture}
\end{center}

\subsection {Teorema de Ramsey}
O teorema de Ramsey generaliza o resultado do problema da festa:

\begin{teorema}
    Para quaisquer $p,q\geq2$ inteiros existe um inteiro $N=R(p,q)$ tal que qualquer coloração de $K_N$ em duas cores possui uma $p$-clique azul ou uma $q$-clique vermelha.
\end{teorema}
\begin{proof}
    Vamos provar por indução em $p$ e $q$.

    Se $p=2$, uma coloração $K_N$ que não contém uma $2$-clique azul deve conter todas as arestas vermelhas. Mas então, para um $N\geq q$, essa coloração possui uma $q$-clique vermelha. Então $R(2,q)=q$. Da mesma forma, vemos que $R(p,2)=p$.

    Para mostrar que $R(p,q)$ existe para qualquer $p$ e $q$, vamos mostrar um limitante superior.

    Sejam $p,q\geq3$ e suponha que exista $R(p,q-1)$ e $R(p,q-1)$. Vamos mostrar que $R(p,q)\leq R(p,q-1)+R(p-1,q)$.

    Seja $N=R(p,q-1)+R(p-1,q)$ e $x$ um vértice de $K_N$. Considere os conjuntos dos vértices que são conectados a $x$ por uma aresta azul e dos que são conectados a $x$ por uma aresta vermelha. Chame-os de $A$ e $V$.

    Claramente $|A|+|V|=N-1=R(p,q-1)+R(p,q-1)-1$. Isso implica que ou $|A|\geq R(p,q-1)-1$ ou $|V|\geq R(p-1,q)-1$, pois caso contrário teríamos $|A|+|V|<R(p,q-1)+R(p-1,q)-2=N-2$.

    Suponha, sem perda de generalidade, que $|A|\geq R(p,q-1)-1$. Considere $K_{|A|}=K_N[A]$ o subgrafo induzido por $A$. Então um dos dois casos vale:
    \begin{itemize}
        \item $K_{|A|}$ contém uma $p$-clique vermelha.
        \item $K_{|A|}$ contém uma $(q-1)$-clique azul.
    \end{itemize}
    Se o primeiro caso for verdade, $K_N$ possui uma $p$-clique vermelha e não há nada a fazer.

    Se o segundo caso for verdade, como todas as arestas de $x$ a vértices de $A$ são azuis, considere os vértices da $(q-1)$-clique azul juntos com $x$. Esses vértices formam uma $q$-clique azul, então $K_N$ possui uma $q$-clique azul.
\end{proof}

\subsection {Generalizando para hipergrafos}
Assim como um grafo, um \textbf{hipergrafo $k$-uniforme} é um par $(V,E)$ em que $V$ é o conjunto de \textbf{vértices} e $E$ é o conjunto de \textbf{arestas}, só que os elementos de $E$ são $k$-subconjuntos de $V$.

O teorema de Ramsey pode ser generalizado da seguinte forma:

\begin{teorema}
    Para quaisquer $k>0$, $p,q\geq k$ inteiros, existe um inteiro $N=R^k(p,q)$ tal que para qualquer coloração das $k$-arestas do hipergrafo $k$-uniforme completo de $N$ vértices existe um conjunto de $p$ vérices tal que todas as $k$-arestas entre eles são azuis ou um conjunto de $q$ vértices tal que todas as $k$-arestas entre eles são vermelhas (alternativamente, para qualquer coloração dos $k$-subconjuntos de $\{1,...,N\}$ existe um $p$-subconjunto cujos $k$-subconjuntos são azuis ou um $q$-subconjunto cujos $k$-subconjuntos são vermelhos).
\end{teorema}
\begin{proof}
    Vamos mostrar por indução em $p$, $q$ e $k$.

    Primeiramente, se $k=1$, é fácil ver que $R^k(p,q)=max\{p,q\}$, já que as colorações são de vértices individuais.

    Agora, se $p=k$, uma coloração que não contém um conjuntp de $p=k$ vértices com as arestas azuis tem que conter todas as arestas vermelhas. Similarmente à base da indução no caso de grafos, temos que $R^k(k,q)=q$ e, por simetria, $R^k(p,k)=p$.

    Assim podemos colocar nossas hipóteses de indução. Sejam $p,q,k$ como no enunciado do teorema. Suponha que existam, para qualquer $x,y$:
    \begin{itemize}
        \item $R^{k-1}(x,y)$
        \item $R^k(p-1,y)$
        \item $R^k(x,q-1)$
    \end{itemize}
    E vamos mostrar que existe $R^k(p,q)$.

    Seja $N=R^{k-1}(R^k(p-1,q),R^k(p,q-1))+1$. Considere $K^N_k$ o hipergrafo $k$-uniforme completo de $N$ vértices e uma coloração $\chi$ de suas arestas.

    Escolha um vértice qualquer $x$ e seja $S$ a coleção de todas as $(k-1)$-arestas que não contém $x$. $S$ define um hipergrafo $(k-1)-uniforme$ completo de $N-1$ vértices e possui uma coloração induzida por $\chi$ dada por $\chi'(e)=\chi(e\cup\{x\})$ para toda aresta $e$ de $S$. 
    
    Pela escolha de $N$, podemos aplicar nossa hipótese de indução em $S$ e colcluir que vale:
    \begin{itemize}
        \item Ou $S$ tem um conjunto $A$ de $R^k(p,q-1)$ vértices tal que toda (k-1) aresta de $S$ em $A$ é azul (para $\chi'$);
        \item Ou $S$ tem um conjunto $V$ de $R^k(p-1,q)$ vértices tal que toda (k-1) aresta de $S$ em $V$ é vermelha (para $\chi'$).
    \end{itemize}
    Sem perda de generalidade, suponha que o primeiro caso que vale.

    Então, pela hipótese de indução, vale uma das seguintes afirmações:
    \begin{itemize}
        \item A tem um $(p-1)$-subconjunto $T$ tal que toda $k$-aresta em $T$ é azul (para $\chi$)
        \item A tem um $q$-subconjunto $T$ tal que toda $k$-aresta em $T$ é vermelha (para $\chi$)
    \end{itemize}
\end{proof}

No segundo caso o teorema está provado. Olhando para o primeiro caso, considere $T'=T\cup\{x\}$. Se uma aresta $a$ em $T'$ contém $x$, ela é azul por que $a\setminus x$ foi pintado de azul em $S$ por $\chi'$. Se uma aresta e $T'$ não contém $x$, então ela também está em $T$ e, por hipótese, é azul. Assim $T'$ é um conjunto de $p$ vértices com todas as arestas nele azuis.

\section{Geometria combinatória}
Vamos dar algumas definições a respeito de conjuntos de pontos no plano. Seja $P$ um conjunto finito de pontos no plano e $conv(P)$ o casco convexo de $P$. Dizemos que $P$ está em \textbf{posição geral} se não existirem três pontos colineares em $P$. 

Dizemos que $P$ está em \textbf{posição convexa} se todo ponto de $P$ estiver na fronteira de $conv(P)$.

Um ponto $v\in P$ é um \textbf{vértice} de $P$ se $conv(P)\neq conv(P\backslash v)$.

Se todos os pontos de $P$ forem vértices, dizemos que $P$ está em \textbf{posição estritamente convexa}.

Um \textbf{$k$-ágono} é o casco convexo de um conjunto de $k$ pontos em posição estritamente convexa.

Se $X$ for um subconjunto de $P$ de $k$ pontos em posição estritamente convexa tal que $conv(X)\cap P=X$, então  $conv(X)$ é um \textbf{$k$-buraco} ou um \textbf{$k$-ágono vazio}.

Uma \textbf{triangulação} de um $k$-ágono $T$ é um conjunto de $3$-ágonos tal que os vértices dos $3$-ágonos também são vértices de $T$, $T$ é a união de todos os $3$-ágonos e a interseção de quaisquer dois $3$-ágonos é vazia ou um segmento de reta.

Chamamos $3$-ágonos de triângulos, $4$-ágonos de quadriláteros, $5$-ágonos de pentágonos, etc.

\subsection{Teorema de Erd\H os-Szekeres}
Um resultado clássico da teoria de Ramsey na geometria combinatória é o seguinte:
\begin{teorema}[Erd\H os-Szekeres, 1935]
    Dado um inteiro positivo $k$, existe um inteiro $ES(k)$ tal que todo conjunto com pelo menos $ES(k)$ pontos em posição geral contém $k$ pontos em posição estritamente convexa.
\end{teorema}

Vamos mostrar uma demonstração dada por Yujia Pan\cite{ES} e, para isso, primeiro vamos resolver um caso particular com $k=5$:
\begin{lema}
    Todo conjunto com $5$ pontos em posição geral contém quatro pontos em posição estritamente convexa.
\end{lema}
\begin{proof}
    Seja $P$ um conjunto com $5$ pontos em posição geral. Se $conv(P)$ for um pentágono ou um quadrado, o lema é trivial.
    Então vamos supor que $conv(P)$ seja um triângulo. Tal triângulo possui dois pontos de $P$, $u$ e $v$, em seu interior. A reta que contém esses dois pontos separa os vértices do triângulo, dois ($p$ e $q$) de um lado e um ($x$) do outro lado. Então $\{u,v,p,q\}$ são quatro pontos em posição estritamente convexa.
\end{proof}

Agora podemos demonstrar o teorema:
\begin{proof}(Teorema de Erd\H os-Szekeres)
    Dado um $k$ inteiro positivo, considere $P$ um conjunto com pelo menos $R^4(k,5)$ pontos em posição geral. Para cada subconjunto $X$ de $P$ de tamanho $4$, vamos colorir ele de vermelho se $conv(X)$ for um quadrilátero e de azul de $conv(X)$ for um triângulo. Pelo teorema de Ramsey para hipergrafos, uma das duas conddições é satisfeita:
    \begin{itemize}
        \item $P$ tem um subconjunto de tamanho $k$ tal que todo subconjunto dele com 4 elementos é vermelho.
        \item $P$ tem um subconjunto de tamanho $5$ tal que todo subconjunto dele com 4 elementos é azul.
    \end{itemize}
    Pelo lema, a segunda opção não é possível, pois ela implicaria que um conjunto com $5$ pontos não tem 4 pontos em posição estritamente convexa. Então existe um subconjunto de tamanho $k$ tal que todo subconjunto dele com $4$ elementos é vermelho. Seja $H$ tal conjunto.

    Vamos mostrar que $H$ está em posição estritamente convexa. Para isso, suponha o contrário. Então existe algum ponto $x$ de $H$ no interior de $conv(H)$. Considere uma triangulação de $conv(H)$. $x$ está dentro de algum triângulo da triangulação. Os três vértices desse triângulo juntos com $x$ formam um conjunto de $4$ pontos que deveria ter sido pintado de azul, o que é uma contradição.

    Logo $H$ é um conjunto de $k$ pontos em posição estritamente convexa.
\end{proof}

\subsection{Teorema de Sylvester-Gallai}
Outro resultado interessante é o teorema de Sylvester-Gallai.
\begin{teorema}   
    Todo contunto finito de pontos tem todos os pontos colineares ou existe uma reta que passa por exatamente dois desses pontos.
\end{teorema}
\begin{proof}

\end{proof}

\chapter{Big-Line-Big-Clique conjecture}
A conjectura seguinte foi proposta por Jan Kára et al em 2005\cite{visibilitygraph}:
\begin{conjectura}\label{conj1}
    Dados dois inteiros $k,l\geq2$, existe um $n=n(k,l)$ tal que todo conjunto finito $P$ com pelo menos $n$ pontos no plano contém ou $k$ pontos visíveis entre si (alternativamente, $\mathcal V(P)$ possui uma $k$-clique) ou $l$ pontos colineares.
\end{conjectura}
Embora seu enunciado seja simples, poucos resultados foram encontrados para ela, ainda estando em aberto para valores de $k$ e $l$ a partir de $6$ e $4$ respectivamente.

Para nos habituarmos com o problema, vamos começar pelos casos mais básicos:

\section{Casos triviais}
Primeiramente, se $k=2$ (ou $l=2$), com certeza todo conjunto com pelo menos dois pontos tem dois pontos visíveis (e dois pontos colineares).

Se $k=3$, ou se tem três pontos visíveis ou todos os pontos no conjunto são colineares, então $n(3,l)=max\{3,l\}$.

Se $l=3$, ou todos os pontos são visíveis para todos os outros ou algum ponto não deixa outros dois se verem, se tendo três pontos colineares. $n(k,3) = max\{k,3\}$.


\section{Caso $k=4$}
Vamos mostrar que os conjuntos que não possuem quatro pontos visíveis possuem muitos pontos colineares. Isto é, para todo $l$ existe um $n$ tal que todo conjunto $P$ com pelo menos $n$ pontos que não possui quatro pontos visívels possui pelo menos $l$ pontos colineares.

Pelo teorema \ref{planork4}, um conjunto que não possui quatro pontos visíveis possui grafo de visibilidade planar. Vamos usar o seguinte lema que caracteriza os grafos de visibilidade planares:

\begin{figure}
    \begin{tikzpicture}
        \node(A) {$(a)$};
        \node[draw, shape=circle] (A1) [right = 0.3cm of A] {};
        \node[draw, shape=circle] (A2) [right = 0.6cm of A1] {};
        \node[draw, shape=circle] (A3) [right = 0.6cm of A2] {};
        \node[draw, shape=circle] (A4) [right = 0.6cm of A3] {};
        \node[draw, shape=circle] (A5) [right = 0.6cm of A4] {};
        \draw (A1)--(A2);
        \draw (A2)--(A3);
        \draw (A3)--(A4);
        \draw (A4)--(A5);

        \node(B) [below = 1.2cm of A] {$(b)$};
        \node[draw, shape=circle] (B1) [right = 0.3cm of B] {};
        \node[draw, shape=circle] (B2) [right = 0.6cm of B1] {};
        \node[draw, shape=circle] (B3) [right = 0.6cm of B2] {};
        \node[draw, shape=circle] (B4) [right = 0.6cm of B3] {};
        \node[draw, shape=circle] (B5) [right = 0.6cm of B4] {};
        \node[draw, shape=circle] (B6) [above = 0.6cm of B3] {};
        \draw (B1)--(B2);
        \draw (B2)--(B3);
        \draw (B3)--(B4);
        \draw (B4)--(B5);
        \draw (B1)--(B6);
        \draw (B2)--(B6);
        \draw (B3)--(B6);
        \draw (B4)--(B6);
        \draw (B5)--(B6);

        \node(C) [below = 1.2cm of B] {$(c)$};
        \node[draw, shape=circle] (C1) [right = 0.3cm of C] {};
        \node[draw, shape=circle] (C2) [right = 0.6cm of C1] {};
        \node[draw, shape=circle] (C3) [right = 0.6cm of C2] {};
        \node[draw, shape=circle] (C4) [right = 0.6cm of C3] {};
        \node[draw, shape=circle] (C5) [right = 0.6cm of C4] {};
        \node[draw, shape=circle] (C6) [above = 0.6cm of C3] {};
        \node[draw, shape=circle] (C7) [below = 0.6cm of C3] {};
        \draw (C1)--(C2);
        \draw (C2)--(C3);
        \draw (C3)--(C4);
        \draw (C4)--(C5);
        \draw (C1)--(C6);
        \draw (C2)--(C6);
        \draw (C3)--(C6);
        \draw (C4)--(C6);
        \draw (C5)--(C6);
        \draw (C1)--(C7);
        \draw (C2)--(C7);
        \draw (C3)--(C7);
        \draw (C4)--(C7);
        \draw (C5)--(C7);

        \node(D) [right = 5.4cm of A] {$(d)$};
        \node (D') [right = 0.3cm of D] {};
        \node[draw, shape=circle] (D1) [right = 0.3cm of D'] {};
        \node[draw, shape=circle] (D2) [right = 0.6cm of D1] {};
        \node[draw, shape=circle] (D3) [right = 0.6cm of D2] {};
        \node[draw, shape=circle] (D4) [right = 0.6cm of D3] {};
        \node[draw, shape=circle] (D5) [right = 0.6cm of D4] {};
        \node[draw, shape=circle] (D6) [above = 0.6cm of D'] {};
        \node[draw, shape=circle] (D7) [below = 0.6cm of D'] {};
        \draw (D1)--(D2);
        \draw (D2)--(D3);
        \draw (D3)--(D4);
        \draw (D4)--(D5);
        \draw (D1)--(D6);
        \draw (D2)--(D6);
        \draw (D3)--(D6);
        \draw (D4)--(D6);
        \draw (D5)--(D6);
        \draw (D1)--(D7);
        \draw (D2)--(D7);
        \draw (D3)--(D7);
        \draw (D4)--(D7);
        \draw (D5)--(D7);
        \draw (D6)--(D7);

        \node(E) [below =2.4cm of D] {$(e)$};
        \node[draw, shape=circle] (E1) [below right = 1cm and 0.3cm of E] {};
        \node[draw, shape=circle] (E2) [right = 3cm of E1] {};
        \node[draw, shape=circle] (E3) [above right = 2.8cm and 1.5cm of E1] {};
        \node[draw, shape=circle] (E4) [below left = 1.1cm and 0cm of E3] {};
        \node[draw, shape=circle] (E5) [below left = 0.9cm and 0cm of E4] {};
        \node[draw, shape=circle] (E6) [below right = 0.5cm and 0.55cm of E4] {};
        \draw (E1)--(E2);
        \draw (E2)--(E3);
        \draw (E3)--(E1);
        \draw (E1)--(E4);
        \draw (E1)--(E5);
        \draw (E2)--(E5);
        \draw (E2)--(E6);
        \draw (E3)--(E6);
        \draw (E3)--(E4);
        \draw (E4)--(E5);
        \draw (E5)--(E6);
        \draw (E6)--(E4);
    \end{tikzpicture}
    \caption{Tipos de grafos de visibilidade planares}
    \label{fig:visibilidadeplanar}
\end{figure}

\begin{lema}\cite{planar}\label{visibilidadeplanar}
    Seja $P$ um conjunto finito de pontos no plano. Então $\mathcal V(P)$ é plano se e somente se satisfaz uma das condições:
    \begin{enumerate}[label=(\alph*)]
        \item
            Todos os pontos de $P$ são colineares.
        \item
            Todos os pontos de $P$ são colineares exceto um.
        \item
            Todos os pontos de $P$ são colineares exceto dois pontos não visíveis.
        \item
            Todos os pontos de $P$ são colieares exceto dois pontos $p$ e $q$ tais que o segmento $\overline{pq}$ não intercepta o menor segmento que contém $P\setminus\{p,q\}$
        \item
            $\mathcal V(P)$ é o grafo $(e)$ desenhado na figura \ref{fig:visibilidadeplanar}
    \end{enumerate}
\end{lema}
\begin{proof}
    Como pode ser visto na figura \ref{fig:visibilidadeplanar} e pode ser facilmente demonstrado, se um grafo de visibilidade satisfaz alguma das condições $(a)-(e)$, ele é planar. Para mostrar a outra direção, seja $P$ um conjunto finito de pontos no plano tal que $\mathcal V(P)$ é planar.

    Se $P$ contém no máximo dois pontos, $P$ satisfaz $(a)$. Então vamos supor que $|P|\geq 3$. Seja $L$ um maior conjunto de pontos colineares em $P$ e $\hat L$ a reta que contém $L$. $|L|\geq 2$. Se $|L|=2$, então $L$ satisfaz $(a)$, $(b)$ ou $(e)$. Vamos supor então que $|L|\geq3$.

    A reta $\hat L$ divide o plano em dois semiplanos. Sejam $S$ e $T$ os conjuntos de vértices em cada um desses semiplanos tal que $|S|\geq|T|$. Se $|S|$ for no máximo 1, então $P$ satisfaz algum de $(a)-(d)$. Então supor que $|S|\geq2$. Seja $v$ um ponto de $S$ mais próximo de $\hat L$ e seja $w$ um ponto de $S\setminus\{v\}$ mais próximo de $\hat L$. Note que a reta que contém $v$ e $w$ não é paralela a $\hat L$, pois, se fosse, para todos $x,y\in L$, as arestas $\overline{vx}$ e $\overline{wy}$ ou as arestas $\overline{vy}$ e $\overline{wx}$ se interceptariam e, como o grafo é planar, existiria um vértice nessa interseção que seria mais próximo de $\hat L$ do que $v$, contradizendo a escolha de $v$. Então a reta que contém $v$ e $w$ intercepta $\hat L$ em um ponto, digamos, $p$.

    Vamos mostrar que em $\hat L$ se tem no máximo um vértice de cada lado de $p$. Para isso, suponha que existam dois vértices $x,y\in L$ tais que $x$ esteja entre $y$ e $p$. Então as arestas $\overline{vy}$ e $\overline{wx}$ se interceptam num ponto mais próximo de $\hat L$ do que $v$ e, como o grafo é planar, deve existir um vértice nessa interseção, contradizendo a escolha de $v$. Mas como $|L|\geq3$, $L$ tem que possuir exatamente $3$ pontos: $p$ e um ponto de cada lado de $p$, que vamos chamar de $x$ e $y$.

    Agora vamos mostrar que $S=\{v,w\}$. Suponha que $S$ contém um terceiro ponto $u$. $u$ não está na mesma reta que $p$, $v$ e $w$ pois $L$ tem tamanho máximo. Pela escolha de $v$ e $w$, $u$ está mais longe de $\hat L$ do que $w$. Então $\overline{uv}$ intercepta $\overline{wx}$ ou $\overline{wy}$ num ponto mais próximo de $\hat L$ do que $w$, contradizendo a escolha de $w$. Logo $|S|=2$. Então $|T|\leq2$.

    Se $T=\emptyset$, então $P$ satisfaz $(c)$. Suponha que $T\neq\emptyset$. Seja $u$ um ponto de $T$ e $q$ a interseção da reta que contém $u$ e $v$ e da reta $\hat L$. Suponha que $q$ não seja um vértice de $L$. Se $q$ está entre $x$ e $y$, então as arestas $\overline{vu}$ e $\overline{px}$ ou $\overline{py}$ se interceptam em $q$, e como o grafo é planar, $q$ tem que ser um vértice do grafo, o que é uma contradição. Similarmente, se $q$ não está entre $x$ e $y$, as arestas $\overline{uv}$ e $\overline{wx}$ ou $\overline{wy}$ se interceptam no ponto $q$, gerando uma contradição. Então $u$ intercepta $\hat L$ num vértice. Esse vértice não pode ser $p$, pois se fosse isso contradiria a escolha de $L$.

    Suponha que existam dois pontos $u_1$ e $u_2$ em $T$. Se $u_1$ está na mesma reta que $v$ e $x$ e $u_2$ está na mesma reta que $v$ e $y$. As arestas $\overline{u_1y}$ e $\overline{u_2x}$ se interceptam em um ponto de $T$ que não está na mesma reta que $v$ e $x$ nem que $v$ e $y$, o que é uma contradição. Se $u_1$ e $u_2$ estão na mesma reta que $v$ e $x$, suponha sem perda de generalidade que $u_1$ está entre $x$ e $u_2$, então as arestas $\overline{u_1y}$ e $\overline{u_2x}$ se interceptam em um ponto de $T$ que não está na mesma reta que $v$ e $x$ nem que $v$ e $y$, o que é uma contradição. Do mesmo jeito, se $u_1$ e $u_2$ estão na mesma reta que $v$ e $y$ também temos uma contradição. Portanto $|T|=1$. Sem perda de generalidade, digamos que $u$ é colinear com $v$ e $x$. Assim $\{p,u,v,x,y,w\}$ forma o grafo $(e)$ da figura \ref{fig:visibilidadeplanar}
\end{proof}
\begin{teorema}
    Seja $P$ um conjunto finito de pontos no plano. As seguintes afirmações são equivalentes:
    \begin{itemize}
        \item
            $P$ satisfaz $(a)$, $(b)$, $(c)$ ou $(e)$ do lema \ref{visibilidadeplanar}.
        \item
            $\mathcal V(P)$ não tem $K_4$.
    \end{itemize}
\end{teorema}
\begin{proof}
    Se $P$ satisfaz $(a)$, $(b)$, $(c)$ ou $(e)$ do lema \ref{visibilidadeplanar}, é fácil ver que $\mathcal V(P)$ não tem $K_4$.

    Se $\mathcal V(P)$ não tem $K_4$, pelo teorema \ref{planork4}, $\mathcal V(P)$ é planar. Logo, pelo lema \ref{visibilidadeplanar}, $P$ satisfaz $(a)$, $(b)$, $(c)$, $(d)$ ou $(e)$. Mas como $(d)$ possui $K_4$, $P$ satisfaz $(a)$, $(b)$, $(c)$ ou $(e)$.
\end{proof}

\begin{corolario}
    A conjectura \ref{conj1} é verdadeira para $k=4$ com $n(4,l)=max\{7,l+2\}$ se $l>3$.
\end{corolario}

\section{Caso $k=5$}
Para resolver esse caso, teremos que estudar quando que um conjunto de pontos em posição convexa tem muitos pontos em posição estritamente convexa com a finalidade de generalizar o teorema de Erd\H os-Szekeres para conjuntos com colinearidades.

\subsection{Pontos em posição convexa}
Considere o seguinte problema: dado um conjunto $P$ de pontos em posição convexa no plano, ache um maior subconjunto de $P$ de pontos em posição estritamente convexa.

Para $k,l$ inteiros positivos, seja $q(k,l)$ o menor inteiro tal que todo conjunto com pelo menos $q(k,l)$ pontos em posição convexa no plano tenha $l$ pontos colineares ou $k$ pontos em posição estritamente convexa.

\begin{lema}\label{convex}
    Para $k,l\geq 3$, 
    $$q(k,l)=
    \begin{cases}
        \frac{1}{2}(k-1)(l-1)+1 \text{ se }k\text{ é ímpar}\\
        \frac{1}{2}(k-2)(l-1)+2  \text{ se }k\text{ é par}
    \end{cases}$$
\end{lema}
\begin{proof}

\end{proof}

Com isso, podemos generalizar o teorema de Erd\H os-Szekeres como foi feito por Abel et al.\cite{pentagon}.

\begin{teorema}\label{EScolinear}
    Para todo inteiro $k$, todo conjunto com $ES(k)$ pontos contém $k$ pontos em posição convexa (não necessariamente estrita).
\end{teorema}
\begin{proof}
    Sejam $P$ e $P'$ dois conjuntos de pontos no plano com cada ponto de $P$ associado a um único ponto de $P'$. Se para todos $v\in P$ e $v'\in P'$ associados vale que $dist(v,v')<\epsilon$, dizemos que $P'$ é uma $\epsilon$-perturbação de $P$.

    Vamos mostrar o seguinte: para todo conjunto de pontos $P$, existe uma $\epsilon$-perturbação $P'$ de $P$ em posição geral para algum $\epsilon>0$ tal que se $S'$ é um subconjunto de $P'$ em posição estritamente convexa então $S\in P$ (associado a $S'$) está em posição convexa.
i
    Daí, o teorema de Erd\H os-Szekeres garante que existe para todo $k$ vai existir um $P'$ com $ES(k)$ pontos e $S'\in P'$ em posição estritamente convexa e assim o teorema estará provado.
    
    Para cada tripla ordenada $(u,v,w)$ de pontos de $P$, existe um $\mu>0$ tal que para todo $0<\epsilon<\mu$ toda $\epsilon$-perturbação mantém a orientação\footnote{A orientação de uma tripla ordenadada de pontos diz se eles são colineares, se viram para a esquerda ou se para a direita.} desses pontos. Como a quantidade de triplas assim são finitas, pegamos $\mu^*$ o $\mu$ mínimo. Seja $P'$ uma $\mu^*$-perturbação de $P$ em posição geral.

    Seja $S'$ um subconjunto de $P'$ em posição estritamente convexa. Considere $S'$ e sentido horário. Então a orientação de toda tripla de pontos seguidos de $S'$ é para a direita. Como tal perturbação preservou a orientação das triplas não colineares, o conjunto $S$ associado a $S'$ só possuia triplas colineares e orientadas para a direita, quando olhadas da mesma ordem que vimos em $S'$.

    Portanto $S$ está em posição convexa.
\end{proof}
\begin{teorema}
    Dados inteiros $k\geq 3$ e $k\geq 2$, existe um inteiro $ES(k,l)$ tal que todo conjunto $P$ com pelo menos $ES(k,l)$ contém 
    \begin{itemize}
        \item $l$ pontos colineares, ou
        \item $k$ pontos em posição estritamente convexa.
    \end{itemize}
\end{teorema}
\begin{proof}
    Basta mostrar que
    $$
    ES(k,l)\leq
    \begin{cases}
        ES(\frac{1}{2}(k-1)(l-1)+1) \text{ se }k\text{ é ímpar}\\
        ES(\frac{1}{2}(k-2)(l-1)+2) \text{ se }k\text{ é par}
    \end{cases}$$
    Se $k$ é ímpar, seja $P$ um conjunto com $ES(\frac{1}{2}(k-1)(l-1)+1)$ pontos sem $l$ pontos colineares. Pelo teorema \ref{EScolinear}, $P$ possui $\frac{1}{2}(k-1)(l-1)+1$ pontos em posição convexa. Assim, pelo lema \ref{convex}, $P$ possui $k$ pontos em posição estritamente convexa.

    Se $k$ é par, a prova é análoga.
\end{proof}

\subsection{Pentágonos vazios}

Agora que sabemos como achar conjuntos em posição estritamente convexa, vamos mostrar o seguinte teorema de Abel et al\cite{pentagon}:

\begin{teorema}
    Para todo inteiro $l\geq 2$, todo conjunto finito com pelo menos $ES(\frac{(2l-1)^l-1}{2l-1})$ pontos no plano possui 
    \begin{itemize}
        \item $l$ pontos colineares, ou
        \item um pentágono vazio.
    \end{itemize}
\end{teorema}
\subsubsection {Rascunho da prova}
Considere um conjunto de pontos $P$. A imagem abaixo mostra o que são as camadas convexas de $P$ (formalmente $A_i$ é o conjunto dos vértices de $(P\cap conv(A_{i-1})) \setminus A_{i-1}$).

\begin{tikzpicture}
    \foreach \point in {(-0.5,0),(5,4),(-1,3),(3,3),(1,-1),(1,-2),(0,-3),(4,1),(0,0),(4,-3),(3,2),(1,2),(1,1), (2,1)}{
        \fill \point circle (2pt);
    }

    \draw[thin] (5,4) -- (4,-3) -- (4,-3) -- (0,-3) -- (0,-3) -- (-1,3) -- (5,4);
    \draw[thin] (3,3) -- (4,1) -- (1,-2) -- (0,0) -- (1,2) -- (3,3);
    \draw[thin] (1,1) -- (3,2) -- (1,-1) -- (1,1);

    \node(A1) at (5,0) {$A_1$};
    \node(A2) at (3,-1) {$A_2$};
    \node(A3) at (2,0) {$A_3$};
\end{tikzpicture}

Vamos supor que um conjunto sulficientemente grande de pontos não tem nenhum pentágono vazio e nem $l$ pontos colineares e achar uma contradição. Para isso, faremos o seguinte:
\begin{enumerate}
    \item Mostrar que tal conjunto tem $l$ camadas convexas não vazias.
    \item Fixar um ponto $z$ que está dentro da $l$-ésima camada convexa.
    \item Mostrar que, para que não tenham pentágonos vazios, muitos pontos tem que se alinhar com $z$ usando um conceito que chamaremos de \textit{arcos vazios}.
\end{enumerate}

\begin{proof}
    Seja $l\geq 3$ e $k=\frac{(2l-1)^l-1}{2l-1}$ e considere um conjunto $P$ com pelo menos $ES(k)$ pontos.

    Suponha que $P$ não contém $l$ pontos colineares nem um pentágono vazio.

    Um conjunto $X\subset P$ em posição convexa com pelo menos $k$ pontos é dito $k$-minimal se não existir outro conjunto $Y\subset P$ em posição convexa com pelo menos $k$ pontos tal que $conv(Y)\subsetneq conv(X)$.

    Pelo teorema \ref{EScolinear}, $P$ possui algum conjunto de $k$ pontos em posição convexa. Seja $A_1$ um conjunto assim $k$-minimal.

    Vamos definir as camadas convexas $A_i$ a partir de $A_1$.
    Para $i\in[l-1]\setminus\{1\}$, seja $A_i$ os vértices de $(P\cap conv(A_{i-1})) \setminus A_{i-1}$ e seja $A_l=(P\cap conv(A_{l-1}))\setminus A_{l-1}$.

    Pelo lema \ref{convex} com $k=5$, para cada $i\in[l-1]\setminus\{1\}$, quaisquer $2l-1$ pontos consecutivos em $A_i$ possuem $5$ pontos em posição estritamente convexa. Para que $P$ não tenha pentágonos vazios, o casco convexo desses $2l-1$ pontos deve conter um ponto de $A_{i-1}$.

    Como $A_{i-1}$ contém $\floor{\frac{|A_{i-1}|}{2l-1}}$ conjuntos distintos de $2l-1$ pontos consecutivos,
    $$|A_i|\geq\floor{\frac{|A_{i-1}|}{2l-1}}>\frac{|A_{i-1}|}{2l-1}-1$$
    E disso concluímos
    $$|A_{i-1}|<(|A_i|+1)(2l-1)$$

    Suponha que para algum $i\in[l-1]$ vale $A_i=\emptyset$. Então $|A_{i-1}| < (2l-1)$, $|A_{i-2}| < (2l-1)^2 + (2l-1)$, $|A_{i-3}| < (2l-1)^3 + (2l-1)^2 + (2l-1)$, ..., e, por indução, 
    $$|A_1|<\sum_{j=1}^{i-1}(2l-1)^j$$
    Somando os elementos da PG, obtemos:
    $$|A_1|=\frac{(2l-1)^i-(2l-1)}{2l-2}<\frac{(2l-1)^i-1}{2l-2}\leq\frac{(2l-1)^l-1}{2l-2}=k$$
    O que é uma conrtadição. Logo $A_i\neq\emptyset, \forall i \in [l]$. Se $|A_i| < 3$, então $A_{i+1}=\emptyset$. Portanto $|A_i|\geq 3,\forall i \in [l-1]$.

    Fixe $z\in A_l$. Para $i$ em $[l-2]$, considere os pontos de $A_i$ em sentido horário. Se $x$ e $y$ são dois pontos consecutivos nessa ordem, dizemos que o segmento orientado $\overrightarrow{xy}$ é um arco vazio se $\Delta (x,y,z)\cap A_{i+1}=\emptyset$.

    Se $\overrightarrow{xy}$ é um arco vazio em $A_i$, $p$ e $q$ são consecutivos em $A_{i+1}$ em sentido horário e $\overrightarrow{pq}$ é o segmento orientado de $A_{i+1}$ que intercepta com $\Delta(x,y,z)$, então dizemos que $\overrightarrow{pq}$ é o arco seguinte a $\overrightarrow{xy}$.

    \begin{tikzpicture}
        \foreach \point in {(-1,2),(1,2),(0,0),(-1,1),(1,1)}{
            \fill \point circle (2pt);
        }

        \draw[thin] (-3,0) -- (-1,2) -> (1,2) -- (3,0);
        \draw[thin] (-2,0) -- (-1,1) -- (1,1) -- (2,0);
        \draw[line width=0.25mm, densely dotted] (0,0) -- (-1,2) -- (1,2) -- (0,0);
        \node(Ai) at (0,1.2) {$A_{i+1}$};
        \node(Ai1) at (0,2.2) {$A_i$};
        \node(x) at (-1.2,2) {$x$};
        \node(y) at (1.2,2) {$y$};
        \node(p) at (-1.2,1) {$p$};
        \node(q) at (1.2,1) {$q$};
        \node(z) at (0.2,0) {$z$};
    \end{tikzpicture}

    \begin{fato}\label{fato1}
        Se $\overrightarrow{xy}$ é um arco vazio de $A_i$ e $\overrightarrow{pq}$ é o arco seguinte a $\overrightarrow{xy}$, então $conv(\{x,y,p,q\})$ é um quadrilátero vazio e $\overrightarrow{pq}$ também é um arco vazio.
    \end{fato}
    \begin{proof}
        Seja $Q=\{x,y,p,q\}$. Como $p,q\in conv(A_i)$, $x$ e $y$ são vértices de $Q$ e como $\overrightarrow{xy}$ é um arco vazio $p$ e $q$ também são vértices de $Q$. Logo $Q$ é um quadrilátero. $Q$ é vazio pela definição das camadas convexas.

        Suponha que $\overrightarrow{pq}$ não seja um arco vazio. Então existe algum ponto em $\Delta{p,q,z}\setminus\Delta(x,y,z)$. Seja $t$ um ponto desses mais próximo a $\overline{pq}$. Como $\{x,y,p,q\}$ é um quadrilátero vazio, então $\{x,y,p,q,t\}$ é um pentágono vazio, o que é uma contradição.

        Logo $\overrightarrow{pq}$ é um arco vazio.
    \end{proof}

    Seja $\overrightarrow{pq}$ o arco seguinte ao arco vazio $\overrightarrow{xy}$, então $\overrightarrow{pq}$ é:
    \begin{itemize}
        \item \textbf{alinhado à esqueda} se $p\in\Delta[x,y,z]$ e $q\notin\Delta[x,y,z]$

        \item \textbf{alinhado à direita} se $p\notin\Delta[x,y,z]$ e $q\in\Delta[x,y,z]$
        \item \textbf{alinhado duplamente} se $p\in\Delta[x,y,z]$ e $q\in\Delta[x,y,z]$
    \end{itemize}
    \begin{tikzpicture}
        \foreach \point in {(-1,2),(1,2),(0,0),(-0.5,1),(1,1)}{
            \fill \point circle (2pt);
        }

        \draw[thin] (-3,0) -- (-1,2) -> (1,2) -- (3,0);
        \draw[thin] (-1.5,0) -- (-0.5,1) -- (1,1) -- (2,0);
        \draw[line width=0.25mm, densely dotted] (0,0) -- (-1,2) -- (1,2) -- (0,0);
        \node(Ai) at (0,1.2) {$A_{i+1}$};
        \node(Ai1) at (0,2.2) {$A_i$};
        \node(x) at (-1.2,2) {$x$};
        \node(y) at (1.2,2) {$y$};
        \node(p) at (-0.7,1) {$p$};
        \node(q) at (1.2,1) {$q$};
        \node(z) at (0.2,0) {$z$};
        \node(label) at (0,-1) {Alinhado à esquerda};
    \end{tikzpicture}
    \begin{tikzpicture}
        \foreach \point in {(-1,2),(1,2),(0,0),(-1,1),(0.5,1)}{
            \fill \point circle (2pt);
        }

        \draw[thin] (-3,0) -- (-1,2) -> (1,2) -- (3,0);
        \draw[thin] (-2,0) -- (-1,1) -- (0.5,1) -- (1.5,0);
        \draw[line width=0.25mm, densely dotted] (0,0) -- (-1,2) -- (1,2) -- (0,0);
        \node(Ai) at (0,1.2) {$A_{i+1}$};
        \node(Ai1) at (0,2.2) {$A_i$};
        \node(x) at (-1.2,2) {$x$};
        \node(y) at (1.2,2) {$y$};
        \node(p) at (-1.2,1) {$p$};
        \node(q) at (0.7,1) {$q$};
        \node(z) at (0.2,0) {$z$};
        \node(label) at (0,-1) {Alinhado à direita};
    \end{tikzpicture}
    \begin{tikzpicture}
        \foreach \point in {(-1,2),(1,2),(0,0),(-0.5,1),(0.5,1)}{
            \fill \point circle (2pt);
        }

        \draw[thin] (-3,0) -- (-1,2) -> (1,2) -- (3,0);
        \draw[thin] (-1.5,0) -- (-0.5,1) -- (0.5,1) -- (1.5,0);
        \draw[line width=0.25mm, densely dotted] (0,0) -- (-1,2) -- (1,2) -- (0,0);
        \node(Ai) at (0,1.2) {$A_{i+1}$};
        \node(Ai1) at (0,2.2) {$A_i$};
        \node(x) at (-1.2,2) {$x$};
        \node(y) at (1.2,2) {$y$};
        \node(p) at (-0.7,1) {$p$};
        \node(q) at (0.7,1) {$q$};
        \node(z) at (0.2,0) {$z$};
        \node(label) at (0,-1) {Duplamente alinhado};
    \end{tikzpicture}
    \begin{tikzpicture}
        \foreach \point in {(-1,2),(1,2),(0,0),(-1,1),(1,1)}{
            \fill \point circle (2pt);
        }

        \draw[thin] (-3,0) -- (-1,2) -> (1,2) -- (3,0);
        \draw[thin] (-2,0) -- (-1,1) -- (1,1) -- (2,0);
        \draw[line width=0.25mm, densely dotted] (0,0) -- (-1,2) -- (1,2) -- (0,0);
        \node(Ai) at (0,1.2) {$A_{i+1}$};
        \node(Ai1) at (0,2.2) {$A_i$};
        \node(x) at (-1.2,2) {$x$};
        \node(y) at (1.2,2) {$y$};
        \node(p) at (-1.2,1) {$p$};
        \node(q) at (1.2,1) {$q$};
        \node(z) at (0.2,0) {$z$};
        \node(label) at (0,-1) {Nenhum};
    \end{tikzpicture}

    \begin{fato}\label{fato2}
        Se $\overrightarrow{pq}$ o arco seguinte ao arco vazio $\overrightarrow{xy}$, então $\overrightarrow{pq}$ é alinhado à esquerda ou alinhado à direita ou alinhado duplamente.
    \end{fato}
    \begin{proof}
        Suponha que $\overrightarrow{pq}$ não seja nenhum desses casos.

        Então seja $t$ o ponto em $D=P\cap(\Delta[p,q,z]\setminus\overline{pq})$ mais próximo a $\overline{pq}$. Tal ponto existe pois $z\in D$ e $D$ é finito.

        Como $\overrightarrow{xy}$ é um arco vazio e $\{x,y,p,q\}$ é um quadrilátero vazio, então $\{x,y,p,q,t\}$ é um pentágono vaio, o que é uma contradição.

        Logo $\overrightarrow{pq}$ é alinhado à esquerda ou alinhado à direita ou alinhado duplamente
    \end{proof}
    Suponha que $A_1$ não contenha arcos vazios. Então para dois pontos consecutivos $x$ e $y$ em $A_1$ temos pelo menos um ponto de $A_2$ em $\Delta(x,y,z)$. Como para todos pares de pontos concecutivos $x,y$ e $u,v$ em $A_1$, $\Delta(x,y,z)\cap\Delta(u,v,z)=\emptyset$, teríamos $|A_2|\geq|A_1|$, o que contradiz a minimalidade de $A_1$.

    Seja $\overrightarrow{x_1y_1}$ um arco vazio em $A_1$ e, para $i\in[l-2]$, seja $\overrightarrow{x_{i+1}y_{i+1}}$ o arco seguinte a $\overrightarrow{x_iy_i}$. Pelo fato \ref{fato1}, $\overrightarrow{x_{i+1}y_{i+1}}$ é um arco vazio.

    Se todos os $\overrightarrow{x_iy_i}$ fossem duplamente alinhados, teríamos $\{x_1,x_2,...,x_{l-1},z\}$ colineares e $\{y_1,y_2,...,y_{l-1},z\}$ colineares e, pelo fato \ref{fato2}, $\{x_1,x_2,...,x_{l-1},x_l,z\}$ ou $\{y_1,y_2,...,y_{l-1},y_l,z\}$ seriam colineares, o que é uma contradição.

    Então algum $\overrightarrow{x_iy_i}$ não á duplamente alinhado. Fixe o menor $i\in[l-2]\setminus\{1\}$ que isso acontece. Pelo fato \ref{fato2}, $\overrightarrow{x_iy_i}$ é alinhado à esquerda ou à direita. Sem perda de generalidade, suponha que é alinhado à esquerda. Existe algum $j\in\{i+1,...l-1\}$ tal que $\overrightarrow{x_jy_j}$ não é alinhado à esquerda, pois caso contrário $\{x_1,x_2,...,x_l,z\}$ seriam colineares. Fixe o menor $j\in\{i+1,...,l-1\}$ que isso acontece.

    Temos que $\{x_{j-2},y_{j-2},x_{j-1},y_{j-1},y_j\}$ estão em posição estritamente convexa. Portanto $P$ contém um pentágono vazio.

    \begin{tikzpicture}
        \foreach \point in {(-1,2),(1,2),(0,0),(0.66666,1),(-0.75,1.5),(1,1.5)}{
            \fill \point circle (2pt);
        }
        \draw[thin] (-3,0) -- (-1,2) -- (1,2) -- (3,0);
        \draw[thin] (-2.25,0) -- (-0.75,1.5) -- (1,1.5) -- (2.5,0);
        \draw[thin] (-0.5,1) -- (0.66666,1) -- (1.5,0);
        \draw[dotted] (-0.8,1) -- (-0.5,1);
        \draw[dotted] (0,0) -- (-1,2);
        \draw[dotted] (0,0) -- (1,2);
        \draw[dotted] (0,0) -- (1,1.5);
        \node(Aj2) at (0,1.2) {$A_{j-2}$};
        \node(Aj1) at (0,1.7) {$A_{j-1}$};
        \node(Aj) at (0,2.2) {$A_j$};
        \node(xj2) at (-1.5,2) {$x_{j-2}$};
        \node(yj2) at (1.5,2) {$y_{j-2}$};
        \node(xj1) at (-1.25,1.5) {$x_{j-1}$};
        \node(yj1) at (1.5,1.5) {$y_{j-1}$};
        \node(yj) at (1,1) {$y_j$};
        \node(z) at (0.2,0) {$z$};
        \draw[opacity=0.5,fill=gray] (1,2) -- (-1,2) -- (-0.75,1.5) -- (0.66666,1) -- (1,1.5);
    \end{tikzpicture}
\end{proof}

Como a melhor cota superior para a ordem de crescimento de $ES(k)$ conhecida é exponencial em $k$\cite{ESbound}, esse teorema nos garante a existência de pentágonos vazios em conjuntos de tamanho muito grandes por conter um exponencial duplo. 

De fato, em 2012 Brát et al\cite{pentagon2} mostraram que o resultado vale para $328l^2$ pontos construindo mais de $l$ camadas convexas com alguma delas com arco vazio e usou o mesmo argumento do teorema a cima.

\begin{corolario}
    A conjectura \ref{conj1} é verdadeira para $k=5$ com $n(5,l)=\Theta(l^2)$.
\end{corolario}
\begin{proof}
    O resultado de Brát et al\cite{pentagon2} garante que $n(5,l)=O(l^2)$. Para ver que $n(5,l)=\Omega(l^2)$ basta ver que uma grade $(l-1)$x$(l-1)$ contém $(l-1)^2$ pontos e não contém pentágonos vazios nem $l$ pontos colineares.
\end{proof}

\section{Para conjuntos infinitos de pontos}
Em 2010\cite{infinity}, Atila Pór e David Wood mostraram que a conjectura \ref{conj1} não vale para conjuntos infinitos de pontos com a seguinte construção:

\begin{teorema}
    Existem conjuntos infinitos enumeráveis de pontos sem 4 pontos colineares e sem 3 pontos visíveis dois a dois.
\end{teorema}
\begin{proof}
    Vamos construir indutivamente um conjunto de pontos com tal propriedade.

    Sejam $x_1$, $x_2$ e $x_3$ três pontos não colineares no plano.

    Dados pontos $x_1,...,x_{n-1}$ não conlineares, vamos definir o ponto $x_n$ da seguinte forma: pelo teorema de Sylvester-Gallai, existe uma reta que passa por exatamente dois pontos de $x_1,...,x_{n-1}$. Seja $\mathcal L=\overleftrightarrow{x_ix_j}$ a reta que passa por exatamente dois pontos de $x_1,...,x_{n-1}$ com $i<j$ e com i mínimo entre as com j mínimo. Coloque $x_n$ no segmento $\overline{x_ix_j}$ tal que ela seja a única reta que contenha $x_n$ e mais dois pontos (isso é possível, já que somente uma quantidade finita de pontos de $\mathcal L$ não pode ser escolhida).

    Vamos chamar de $P = \{x_i|i\in \mathbb N\}$. Por construção, $P$ não tem 4 pontos colineares e se $x_i$ e $x_k$ com $i<k$ são visíveis então existe um outro ponto colinear $x_j$ com $j<k$ e $x_k$ está no segmento $\overline{x_ix_j}$.

    Suponha que $x_i$, $x_j$ e $x_k$ são dois a dois visíveis com $i<j<k$. Então existem pontos $x_{i'}$ e $x_{j'}$ com $i'<k$ e $j'<k$ tais que $x_k$ está no segmento $\overline{x_ix_{i'}}$ e no segmento $\overline{x_jx_{j'}}$. Pela escolha de $x_k$, só existe uma reta que contém $x_k$ e mais dois pontos entre $x_1,...x_{k-1}$. Então $i=j'$ e $j=i'$. Mas então $P\cap\overline{x_ix_j}=\{x_i,x_j,x_k\}$, ou seja, $x_i$ e $x_j$ não são visíveis, ou seja, $x_i$ e $x_j$ não são visíveis.

    Portanto não existem 3 pontos visíveis dois a dois em $P$.

\end{proof}

\section{Dificuldades encontradas}
\subsection{The orchard problem e o grafo de Turán}
Problema citado em \cite{visblock}, solução do orchard problem tem menos arestas do que o grafo de Turán para $k\geq 5$

\subsection{Conjuntos de pontos sem heptágonos vazios}
Dado que o teorema de Erd\H os-Szekeres vale, Erd\H os colocou o problema de determinar se para um dado $n$ existe um $g(n)$ tal que todo conjunto com $g(n)$ pontos em posição geral contém um $n$-ágono vazio.

Se essa questão fosse respondida afirmativamente, a conjectura \ref{conj1} sairia como corolário.

Para $n=3$ e $n=4$ é fácil ver que $g(3)=3$ e $g(4)=5$. Harborth\cite{Harborth1978} provou que $g(5)=10$ e Gerken\cite{Gerken} e Nicolas\cite{Nicolas} provaram que $g(6)$ existe.

No entanto, Horton\cite{heptagon} mostrou uma constrção de conjuntos arbitrariamente grandes sem heptágonos vazios.

\begin{teorema}
    Existem conjuntos arbitrariamente grandes sem heptágonos.
\end{teorema}
\begin{proof}

\end{proof}
\chapter{Bloqueadores de visibilidade}

\section{Ordem de crescimento de $b(n)$}
\cite{block,blockers}

\section{Conjuntos de pontos em posição convexa}
\cite{block,blockers}


\bibliographystyle{plain}
\bibliography{../ref}{}

\end{document}

