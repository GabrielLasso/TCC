%% ------------------------------------------------------------------------- %%
\chapter{Introdução}
\label{cap:introducao}

\section{Motivação}
A teoria de Ramsey é um campo da combinatória que estuda questôes como "quão grande uma estrutura deve ser para que ela possua uma subestrutura com uma certa propriedade?". Um exemplo clássico é o problema da festa: quantas pessoas tem que ter em uma festa para que nela existam ou três pessoas que se conhecem ou três pessoas que não se conhecem entre si. Nesse caso sabe-se que bastam 6 pessoas. Essa teoria possui aplicações em diversos outros campos da matemática, como a topologia, a teoria ergódica e a geometria (\cite{ES}).

Aqui fazemos um estudo de um problema geométrico ligado a teoria de Ramsey. Tal problema foi colocado em 2005 por Jan Kára, Attila Pór e David R. Wood (\cite{visibilitygraph}) enquanto estudavam o número cromático de grafos de visibilidade. O problema pode ser colocado da seguinte forma: "Quão grande deve ser um conjunto de pontos para que ele conhtenhai ou $l$ pontos colineares ou $k$ pontos visíveis dois a dois?".

Uma observação sobre esse problema é que se dois pontos não são visíveis, então existe um terceiro que bloqueia a visibilidade deles. Outro problema estudado aqui é sobre o tamanho mínimo do conjunto de bloqueadores para um conjunto de pontos.

Todos esses conceitos serão definidos com mais formalidade mais pra frente. O primeiro problema é conhecido como \textbf{conjectura big-line-big-clique} e o segundo é uma questão natural sobre \textbf{bloqueadores de visibilidade}.

\section{Estrutura do trabalho}
No capítulo 2 vamos definir alguns conceitos que serão base para os demais, assim como demonstrar algumas propriedades chaves desses conceitos. No capítulo 3 vamos mostrar os resultados conhecidos na literatura sobre a conjectura big-line-big-clique, mostrar algumas dificuldades encontradas na tentativa de resolver esse problema e discutir alguns fatos sobre ele. No capítulo 4 vamos focar no problema dos bloqueadores de visibilidade, mostrando cotas superiores e inferiores conhecidas e discutindo algumas coisas em aberto que as pessoas acreditam que sejam verdade.

