\documentclass[a4paper]{article}
\usepackage[brazilian]{babel}
\usepackage[utf8]{inputenc}
\usepackage[T1]{fontenc}
\usepackage{amsthm}
\usepackage[colorlinks=true, allcolors=blue]{hyperref}

\title{Big Line Big Clique conjecture e bloqueadores de visibilidade}
\author{Gabriel K. Lasso}
\date{}

\newtheorem*{bigline}{Big Line Big Clique conjecture}
\newtheorem*{bloqueadores}{Bloqueadores de visibilidade}

\begin{document}
\maketitle
\section{Motivação}
No estudo de grafos de visibilidade, em 2005, Kára, Pór e Wood\cite{visibilitygraph} conjecturaram o seguinte:
\begin{bigline}
Para inteiros $k, l\geq 2$ existe um $n$ tal que todo conjunto $P$ com pelo menos $n$ pontos contém $l$ pontos colineares ou $k$ pontos visíveis dois a dois (equivalentemente, o grafo de visibilidade de $P$ tem um $k$-clique).
\end{bigline}
Esse problema chamou bastante a atenção de pesquisadores na época\cite{pentagon, visblock, infinity} e ainda permanece em aberto, com não muitos resultados.

Porém, essa curiosidade levou pesquisadores a estudar conjuntos de bloqueadores de visibilidades, o que gerou uma outra questão também tão interessante quanto:
\begin{bloqueadores}
Seja $P$ um conjunto de $n$ pontos no plano sem três pontos colineares. Dizemos que um conjunto $Q$ bloqueia $P$ se $P\cap Q=\emptyset$ e todo segmento com extremidades em $P$ contém um ponto em $Q$. $b(P)$ é o tamanho do menor conjunto que bloqueia $P$ e $b(n)$ é o mínimo de todos os $b(P)$s para todos os conjuntos $P$s de n elementos.
\end{bloqueadores}
Existem alguns resultados sobre a ordem de crescimento de $b(n)$, mas ainda há muita coisa a ser resolvida nessa área.

\section{Objetivo}
Nesse trabalho, eu pretendo fazer um estudo rigoroso sobre os resultados existentes nos problemas cidados a cima e possivelmente em alguns tópicos relacionados, em particular, provando os seguintes resultados:
\begin{itemize}
\item
    Big Line Big Clique conjecture é verdadeira para $k=4$\cite{visibilitygraph}.
\item
    Também é para $k=5$\cite{pentagon}.
\item
    É falsa para conjuntos infinitos de pontos\cite{infinity}.
\item
    Existe $c$ tal que $b(n)\leq 2^{c\sqrt{logn}}$\cite{blockers}.
\item
    Para todo conjunto $P$ de $n$ pontos em posição estritamente convexa, $b(P)$ é $\Omega(nlogn)$\cite{blockers}.
\item
    $b(n)\geq(\frac{25}{8}-o(1))n$\cite{block}.
\end{itemize}

\bibliographystyle{plain}
\bibliography{../ref}{}

\end{document}
