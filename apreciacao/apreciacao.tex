\documentclass{article}

\title{Apreciação pessoal}
\author{Gabriel Kuribara Lasso}
\date{}

\begin{document}
\maketitle
\section*{O curso}
Nos últimos quatro anos da minha vida, estive no curso de Bacharelado em Ciência da Computação, no qual passei por várias experiências, aprendi muitas coisas e me desenvolvi bastante.

Um aspecto muito amigável do curso é a liberdade que os alunos tem, principalmente nos dois últimos anos. Com várias matérias optativas, o aluno pode escolher qual a área da computação que mais lhe agrada, usando, se quiser, uma das trilhaspré definidas como guia para a escolha das matérias.

No entanto, talvez por eu ter cursado a matéria MAC0350 em seu primeiro oferecimento, eu senti que faltaram alguns pontos importantes a serem cobertos nas matérias obrigatórias, como uma matéria de bancos de dados e uma de técnicas de programação (mesmo esses dois assuntos sendo cobertos por MAC0350, eu acho que todos eles dados em uma matéria fica muito denso e não consegui absorver direito). Acho que se MAC0350 fosse substituído por mais matérias, o curso teria um proveito melhor.

Outra matéria que não vi muita utilidade foi FLC0474. Reconheço que seja importante saber redigir textos com qualidade, mas ter uma matéria para isso não gera interesse nos alunos de computação. Ao meu ver, se os professores pedissem para os alunos escreverem relatórios em algumas outras matérias, FLC0474 não seria necessária.

No geral, me arrependo de não ter pego algumas matérias, mas são tantas matérias interessantes que não é possível cursar todas. O curso de Bacharelado em Ciência da Computação é um ótimo curso que ensina vários aspectos, tanto teóricos quanto práticos, de computação.

\section*{O trabalho}
No início da disciplina MAC0499 eu não tinha ideia de que tema desenvolver no trabalho de conclusão de curso. Como eu cursei as matérias de Otimização Combinatória, Algoritmos em Grafos e Introdução a Teoria dos Grafos e gostei delas, eu decidi procurar algo relacionado, foi quando eu conversei com o professor Carlinhos e ele sugeriu o meu tema.

Foi bem prazeroso estudar o problema sugerido pelo professor, assim como outros problemas relacionados. No entanto, algumas partes mais difíceis deram bastante trabalho para serem compreendidas. Mas todos os desafios foram importantes para o meu aprendizado.

Apesar de ter que estudar vários assuntos de matemática discreta na elaboração do trabalho, o maior aprendizado foi adquirido pela experência de desenvolver um projeto grande, o que não se é experienciado nas demais matérias do curso. Aprender a planejar o escopo do trabalho sem um enunciado que te diga o que fazer, a correr atrás do conhecimento necessário para realizá-lo, a ter auto-disciplina para produzí-lo sem deixar para a última hora, tudo isso foi muito importante para o desenvolvimento do trabalho.

Ao longo do desenvolvimento do trabalho, o meu orientador sempre esteve presente e se preocupou em como as coisas estavam indo. A ele, sou muito grato pela ajuda e preocupação que sempre tem com todos os alunos do curso.

\end{document}
